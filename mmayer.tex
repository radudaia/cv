%%%%%%%%%%%%%%%%%
% This is an sample CV template created using altacv.cls
% (v1.7, 9 Aug 2023) written by LianTze Lim (liantze@gmail.com), based on the
% CV created by BusinessInsider at http://www.businessinsider.my/a-sample-resume-for-marissa-mayer-2016-7/?r=US&IR=T
%
%% It may be distributed and/or modified under the
%% conditions of the LaTeX Project Public License, either version 1.3
%% of this license or (at your option) any later version.
%% The latest version of this license is in
%%    http://www.latex-project.org/lppl.txt
%% and version 1.3 or later is part of all distributions of LaTeX
%% version 2003/12/01 or later.
%%%%%%%%%%%%%%%%

%% Use the "normalphoto" option if you want a normal photo instead of cropped to a circle
% \documentclass[10pt,a4paper,normalphoto]{altacv}

\documentclass[10pt,a4paper,ragged2e,withhyper]{altacv}
%% AltaCV uses the fontawesome5 package.
%% See http://texdoc.net/pkg/fontawesome5 for full list of symbols.

% Change the page layout if you need to
\geometry{left=1.25cm,right=1.25cm,top=1.5cm,bottom=1.5cm,columnsep=1.2cm}

% The paracol package lets you typeset columns of text in parallel
\usepackage{paracol}


% Change the font if you want to, depending on whether
% you're using pdflatex or xelatex/lualatex
% WHEN COMPILING WITH XELATEX PLEASE USE
% xelatex -shell-escape -output-driver="xdvipdfmx -z 0" mmayer.tex
\ifxetexorluatex
  % If using xelatex or lualatex:
  \setmainfont{Lato}
\else
  % If using pdflatex:
  \usepackage[default]{lato}
\fi

% Change the colours if you want to
\definecolor{VividPurple}{HTML}{3E0097}
\definecolor{SlateGrey}{HTML}{2E2E2E}
\definecolor{LightGrey}{HTML}{666666}
% \colorlet{name}{black}
% \colorlet{tagline}{PastelRed}
\colorlet{heading}{VividPurple}
\colorlet{headingrule}{VividPurple}
% \colorlet{subheading}{PastelRed}
\colorlet{accent}{VividPurple}
\colorlet{emphasis}{SlateGrey}
\colorlet{body}{LightGrey}

% Change some fonts, if necessary
% \renewcommand{\namefont}{\Huge\rmfamily\bfseries}
% \renewcommand{\personalinfofont}{\footnotesize}
% \renewcommand{\cvsectionfont}{\LARGE\rmfamily\bfseries}
% \renewcommand{\cvsubsectionfont}{\large\bfseries}

% Change the bullets for itemize and rating marker
% for \cvskill if you want to
\renewcommand{\cvItemMarker}{{\small\textbullet}}
\renewcommand{\cvRatingMarker}{\faCircle}
% ...and the markers for the date/location for \cvevent
% \renewcommand{\cvDateMarker}{\faCalendar*[regular]}
% \renewcommand{\cvLocationMarker}{\faMapMarker*}


% If your CV/résumé is in a language other than English,
% then you probably want to change these so that when you
% copy-paste from the PDF or run pdftotext, the location
% and date marker icons for \cvevent will paste as correct
% translations. For example Spanish:
% \renewcommand{\locationname}{Ubicación}
% \renewcommand{\datename}{Fecha}


%% Use (and optionally edit if necessary) this .tex if you
%% want to use an author-year reference style like APA(6)
%% for your publication list
% \input{pubs-authoryear.tex}

%% Use (and optionally edit if necessary) this .tex if you
%% want an originally numerical reference style like IEEE
%% for your publication list
%\input{pubs-num.tex}

%% sample.bib contains your publications
%\addbibresource{sample.bib}

\begin{document}
\name{Radu Daia}
\tagline{Software Engineer}
% Cropped to square from https://en.wikipedia.org/wiki/Marissa_Mayer#/media/File:Marissa_Mayer_May_2014_(cropped).jpg, CC-BY 2.0
%% You can add multiple photos on the left or right
%photoR{2.5cm}{mmayer-wikipedia-cc-by-2_0}
% \photoL{2cm}{Yacht_High,Suitcase_High}
\personalinfo{%
  % Not all of these are required!
  % You can add your own with \printinfo{symbol}{detail}
  \email{radu.daia@pm.me}
%   \phone{000-00-0000}
  %\mailaddress{Address, Street, 00000 County}
  %\location{Sunnyvale, CA}
  \homepage{rdaia.ro}
  %\twitter{@marissamayer}
  \linkedin{radu-daia}
   \github{radudaia} % I'm just making this up though.
%   \orcid{0000-0000-0000-0000} % Obviously making this up too.
  %% You can add your own arbitrary detail with
  %% \printinfo{symbol}{detail}[optional hyperlink prefix]
  % \printinfo{\faPaw}{Hey ho!}
  %% Or you can declare your own field with
  %% \NewInfoFiled{fieldname}{symbol}[optional hyperlink prefix] and use it:
  % \NewInfoField{gitlab}{\faGitlab}[https://gitlab.com/]
  % \gitlab{your_id}
	%%
  %% For services and platforms like Mastodon where there isn't a
  %% straightforward relation between the user ID/nickname and the hyperlink,
  %% you can use \printinfo directly e.g.
  % \printinfo{\faMastodon}{@username@instace}[https://instance.url/@username]
  %% But if you absolutely want to create new dedicated info fields for
  %% such platforms, then use \NewInfoField* with a star:
  % \NewInfoField*{mastodon}{\faMastodon}
  %% then you can use \mastodon, with TWO arguments where the 2nd argument is
  %% the full hyperlink.
  % \mastodon{@username@instance}{https://instance.url/@username}
}

\makecvheader

%% Depending on your tastes, you may want to make fonts of itemize environments slightly smaller
\AtBeginEnvironment{itemize}{\small}

%% Set the left/right column width ratio to 6:4.
\columnratio{0.6}

% Start a 2-column paracol. Both the left and right columns will automatically
% break across pages if things get too long.
\begin{paracol}{2}

\cvsection{Experience}

\cvevent{Senior Embedded Software Engineer}{Tremend Software Consulting}{August 2022 -- Ongoing}{Bucharest, RO}
Projects:
\begin{itemize}
\item Secure IoT Platform (internal project)
\item \href{https://www.aristocratgaming.com/us/}{Aristocrat Gaming} [Nov 2022 - Jan 2023]
\item Automotive Linux BSP (\href{https://www.nxp.com/products/processors-and-microcontrollers/s32-automotive-platform/s32g-vehicle-network-processors:S32G-PROCESSORS}{NXP Semiconductors}) [Feb 2023 - ongoing]
\end{itemize}

\divider

\cvevent{Teaching Assistant}{Polytechnic University of Bucharest}{Feb 2017 -- Ongoing}{Bucharest, RO}
Subjects:
\begin{itemize}
\item Computer Systems Architecture
\item Parallel Architecture Programming
\end{itemize}

\divider

\cvevent{R\&D Senior Software Engineer}{Keysight}{August 2021 -- July 2022}{Bucharest, RO}
\begin{itemize}
    \item \href{https://www.keysight.com/us/en/products/network-test/protocol-load-test/ixload.html}{ixLoad} project
\end{itemize}

\divider

\cvevent{Senior Software Engineer \& Technical Product Owner}{Harman}{Dec 2015 -- August 2021 }{Bucharest, RO}
Automotive infotainment projects development:
\begin{itemize}
    \item BMW NBTevo
    \item BMW MGU 2018 / 2021
    \item BMW MGU 2022 / IDC 2023 / IDC 2024    
\end{itemize}
Additionally, participated in RFQs/RFIs for VW/AUDI and Daimler.
\\
\divider

\cvevent{Software Engineer}{JLG Consulting}{Mar 2015 -- Nov 2015}{Bucharest, RO}

\begin{itemize}
\item \href{https://www.eurocontrol.int/tool/toolkit-air-traffic-management-occurrence-investigation}{TOKAI} project for Air Traffic Management
\end{itemize}

% \divider

% \cvevent{Product Engineer}{Google}{23 June 1999 -- 2001}{Palo Alto, CA}

% \begin{itemize}
% \item Joined the company as employe \#20 and female employee \#1
% \item Developed targeted advertisement in order to use user's search queries and show them related ads
% \end{itemize}

\cvsection{Certifications}

\cvevent{KNX Partner}{KNX Association}{Jan 2024}{}

\cvevent{iSAQB Certified Professional for Software Architecture - Foundation Level® (CPSA-F®)}{International Software Architecture Qualification Board (iSAQB)}{Oct 2023}{}

%\divider

% use ONLY \newpage if you want to force a page break for
% ONLY the currentc column
%\newpage

%\cvsection{Publications}

%% Specify your last name(s) and first name(s) as given in the .bib to automatically bold your own name in the publications list.
%% One caveat: You need to write \bibnamedelima where there's a space in your name for this to work properly; or write \bibnamedelimi if you use initials in the .bib
%% You can specify multiple names, especially if you have changed your name or if you need to highlight multiple authors.
%\mynames{Lim/Lian\bibnamedelima Tze,
%  Wong/Lian\bibnamedelima Tze,
%  Lim/Tracy,
%  Lim/L.\bibnamedelimi T.}
%% MAKE SURE THERE IS NO SPACE AFTER THE FINAL NAME IN YOUR \mynames LIST

%\nocite{*}

%\printbibliography[heading=pubtype,title={\printinfo{\faBook}{Books}},type=book]

%\divider

%\printbibliography[heading=pubtype,title={\printinfo{\faFile*[regular]}{Research}}, type=article]

%\divider

%\printbibliography[heading=pubtype,title={\printinfo{\faUsers}{Conference Proceedings}},type=inproceedings]

%% Switch to the right column. This will now automatically move to the second
%% page if the content is too long.
\switchcolumn

%\cvsection{Life Philosophy}
%\begin{quote}
%``If you don't have any shadows, you're not standing in the light.''
%\end{quote}

\cvsection{Skills}

\cvtag{software engineering}
\cvtag{data structures}
\cvtag{algorithms}
\cvtag{software architecture}\\
\cvtag{computer architecture}
\cvtag{embedded systems}
\cvtag{x86}
\cvtag{ARM}
\cvtag{system design}
\cvtag{automotive}
\cvtag{networking}
\cvtag{operating systems}
\cvtag{data analysis}
\cvtag{scientific research}

\divider\smallskip

\cvtag{C}
\cvtag{Python}
\cvtag{Linux}
\cvtag{Yocto}
\cvtag{Bash}
\cvtag{CMake}
\cvtag{C++}
\cvtag{C\#}
\cvtag{\LaTeX}
\cvtag{Assembly}
\cvtag{git}
\cvtag{CUDA}
\cvtag{MPI}
\cvtag{Apptainer/Singularity}
\cvtag{Matlab}
\cvtag{SQL}
\cvtag{KNX}

\divider\smallskip

\cvtag{product management}
\cvtag{project management}
\cvtag{teaching}
\cvtag{public speech}
\cvtag{open source}

\cvsection{Languages}
\cvskill{Romanian}{5}
\cvskill{English}{5}
\cvskill{French}{3}
\cvskill{German}{1.5} %% supports X.5 values.

\cvsection{Education}

\cvevent{PhD Student in Computer Science \& Engineering}{Polytechnic University of Bucharest}{Oct 2021 -- Ongoing}{}

\divider\smallskip

\cvevent{CERN School of Computing Alumnus}{CERN}{Sept 2022}{Krakow, PL}
\begin{itemize}
    \item Software Engineering \& Computational Physics
\end{itemize}

\divider\smallskip

\cvevent{B.S. \& M.S.\ in Computer Science \& Engineering}{Polytechnic University of Bucharest}{Oct 2012 -- Sept 2018}{}


\end{paracol}

\end{document}
